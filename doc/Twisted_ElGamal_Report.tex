\newcommand{\path}{/Users/chenyu/Documents/Research/PaperEngine}
\documentclass[a4paper,10pt]{article}
\usepackage{authblk}

\input{\path/mypapersetting}
\usepackage{float}

\newcommand{\Oracle}{}


\begin{document}


\thispagestyle{empty}

\title{Twisted ElGamal}

\author{Yu Chen\\ {\normalsize \blue{\texttt{cycosmic@gmail.com}}}}

\affil{Shandong University}


\date{}

\maketitle

\begin{abstract}
In this work, we introduce a new additively homomorphic public-key encryption scheme called twisted ElGamal,  
which is not only as secure and efficient as standard exponential ElGamal, 
but also very friendly to zero-knowledge proof protocols, such as Sigma protocols and range proofs.    
This feature makes twisted ElGamal extremely useful in numerous privacy-preserving scenarios, 
such as cryptocurrencies with strong privacy and secure machine learning.      
% \begin{trivlist}
% \item \textbf{Keywords:} cryptocurrencies, confidential transaction, accountability, twisted ElGamal 
% \end{trivlist}
\end{abstract}

\thispagestyle{empty}

\setcounter{tocdepth}{2}
\tableofcontents
\newpage

\pagestyle{plain}
\pagenumbering{arabic}
\setcounter{page}{1}

\section{Introduction}
In this report, we introduce twisted ElGamal~\cite{CMT2019}, 
which is additively homomorphic and friendly to zero-knowledge proofs.
 

\section{Preliminaries}
We first fix some notations and recall the related cryptographic assumptions. 
\subsection{Basic Notations}
We denote the security parameter by $\lambda \in \mathbb{N}$. 
A function is negligible in $\lambda$, written $\mathsf{negl}(\lambda)$, 
if it vanishes faster than the inverse of any polynomial in $\lambda$.
Let $X = \{X_\lambda\}_{\lambda \in \mathbb{N}}$ and $Y = \{Y_\lambda\}_{\lambda \in \mathbb{N}}$ 
be two distribution ensembles indexed by $\lambda$. 
We say that $X$ and $Y$ are statistically indistinguishable, written $X \approx_s Y$, 
if the statistical distance between $X_\lambda$ and $Y_\lambda$ is negligible in $\lambda$. 
We say that $X$ and $Y$ are computationally indistinguishable, written $X \approx_c Y$, 
if the advantage of any PPT algorithm in distinguishing $X_\lambda$ and $Y_\lambda$ is $\mathsf{negl}(\lambda)$. 
A probabilistic polynomial time (PPT) algorithm is a randomized algorithm that runs in time $\mathsf{poly}(\lambda)$.
If $\mathcal{A}$ is a randomized algorithm, we write $z \leftarrow \mathcal{A}(x_1, \dots, x_n;r)$
to denote that $\mathcal{A}$ outputs $z$ on inputs $(x_1, \dots, x_n)$ and randomness $r$. 
For notational clarity we usually omit $r$ and write $z \leftarrow \mathcal{A}(x_1, \dots, x_n)$.  
For a positive integer $n$, we use $[n]$ to denote the set of numbers $\{1, \dots, n\}$.
For a set $X$, we use $|X|$ to denote its size 
and use $x \sample X$ to denote sampling $x$ uniformly at random from $X$. 
We use $U_X$ to denote the uniform distribution over $X$. 

\subsection{Cryptographic Assumptions}
Let $\mathsf{GroupGen}$ be a PPT algorithm that on input a security parameter $1^\lambda$, 
outputs description of a cyclic group $\mathbb{G}$ of prime order $p = \Theta(2^\lambda)$, 
and a random generator $g$ for $\mathbb{G}$. 
In what follows, we describe the discrete-logarithm based assumptions related to 
$(\mathbb{G}, g, p) \leftarrow \mathsf{GroupGen}(1^\lambda)$.   

\begin{definition}[Decisional Diffie-Hellman Assumption]
The DDH assumption holds if for any PPT adversary, we have: 
\begin{equation*}
\left|\Pr[\mathcal{A}(g, g^a, g^b, g^{ab}) = 1] - 
\Pr[\mathcal{A}(g, g^a, g^b, g^c) = 1] \right| \leq \mathsf{negl}(\lambda),
\end{equation*}
where the probability is over the randomness of $\mathsf{GroupGen}(1^\lambda)$, $\mathcal{A}$'s random tape, 
and the random choices of $a, b, c \sample \mathbb{Z}_p$. 
\end{definition}

\begin{definition}[Divisible Decisional Diffie-Hellman Assumption]
The divisible DDH assumption holds if for any PPT adversary, we have: 
\begin{equation*}
\left|\Pr[\mathcal{A}(g, g^a, g^b, g^{a/b}) = 1] - \Pr[\mathcal{A}(g, g^a, g^b, g^c) = 1] \right| 
\leq \mathsf{negl}(\lambda),
\end{equation*}
where the probability is over the randomness of $\mathsf{GroupGen}(1^\lambda)$, $\mathcal{A}$'s random tape,  
and the random choices of $a, b, c \sample \mathbb{Z}_p$.  
\end{definition}
As proved in~\cite{BDZ2003}, the divisible DDH assumption is equivalent to the standard DDH assumption.
 

\subsection{Public-Key Encryption}
A PKE scheme consists of four polynomial time algorithms as follows. 
\begin{itemize} \itemsep 1pt \parskip 0pt \parsep 0pt
\item $\mathsf{Setup}(1^\lambda)$: on input a security parameter $1^\lambda$, 
    output public parameters $pp$. We assume that $pp$ includes the descriptions of message space $M$, 
    randomness space $R$. 

\item $\mathsf{KeyGen}(pp)$: on input $pp$, output a public key $pk$ and a secret key $sk$.

\item $\mathsf{Enc}(pk, m)$: on input a public key $pk$ and a plaintext $m$, output a ciphertext $C$. 
    When emphasizing the randomness $r$ used for encryption, we denote this by $C \leftarrow \mathsf{Enc}(pk, m; r)$.   

\item $\mathsf{Dec}(sk, C)$: on input a secret key $sk$ and a ciphertext $C$, 
    output a plaintext $m$ or a distinguished symbol $\bot$ indicating that $C$ is invalid.
\end{itemize}

\begin{remark}
In practice, it is common for public parameters to be generated and fixed ``once-and-for-all'', 
and then shared by multiple users. Of course, each user can generate its own public parameters. 
In this case, $\mathsf{Setup}$ is absorbed into $\mathsf{KeyGen}$ and $pp$ becomes a part of public key. 
\end{remark}

\begin{trivlist}
\item \textbf{Correctness.} For all $pp \leftarrow \mathsf{Setup}(1^\lambda)$, 
    all $(pk, sk) \leftarrow \mathsf{KeyGen}(pp)$ and all $m \in M$ (here $M$ is the message space), 
    we have $\mathsf{Dec}(sk, \mathsf{Enc}(pk, m)) = m$.
\end{trivlist}

\begin{trivlist}
\item \textbf{Homomorphism.} For any public key $pk$, any $(m_1, r_1), (m_2, r_2) \in M \times R$, 
    we have $\mathsf{Enc}(pk, m_1; r_1) + \mathsf{Enc}(pk, m_2; r_2) = \mathsf{Enc}(pk, m_1+m_2; r_1+r_2)$. 
    Here, we slight abuse the symbol ``+'' to denote component-wise operation over ciphertext space $\mathbb{G}^2$. 
    Additive homomorphism on message already suffices for most applications: 
    $\mathsf{Enc}(pk, m_1) + \mathsf{Enc}(pk, m_2)$ is an encryption of $(m_1+m_2)$ of $pk$ under some appropriate randomness. 
    When both ciphertext and message spaces are finite cyclic groups, we can naturally define scalar multiplication, 
    which is a special case of additive homomorphism: 
    for all $k \in \mathbb{Z}_p$, $k \cdot \mathsf{Enc}(pk, m)$ is an encryption of $km$ of $pk$. 
\end{trivlist}

For clarity, we define extended algorithms of PKE as below: 
\begin{itemize} \itemsep 1pt \parskip 0pt \parsep 0pt
    \item $\mathsf{ReRand}(sk, C; r)$: on input $sk$ and $c$, output a re-randomized ciphertext $C'$ of $C$. 

    \item $\mathsf{HomoAdd}(C_1, C_2)$: on input $C_1$ and $C_2$, output $C = C_1 + C_2$.  

    \item $\mathsf{HomoSub}(C_1, C_2)$: on input $C_1$ and $C_2$, output $C = C_1 - C_2$. 

    \item $\mathsf{ScalarMul}(C, k)$: on input $C$ and a scalar $k \in \mathbb{Z}_p$, output $C' = k \cdot C$.  
\end{itemize}


\begin{trivlist}
\item \textbf{IND-CPA security.} Let $\mathcal{A}$ be an adversary against the semantic security of 
PKE component and define its advantage in the following experiment:
\begin{displaymath}
    \AdvA = \Pr \left[\beta=\beta': 
    \begin{array}{ll}
        & pp \leftarrow \mathsf{Setup}(1^\lambda); \\
        & (pk, sk) \leftarrow \mathsf{KeyGen}(pp)\\
        & (state, m_0, m_1) \leftarrow \mathcal{A}_1(pp, pk);\\
        & \beta \sample \{0,1\};\\
        & C^* \leftarrow \mathsf{Enc}(pk, m_\beta)\\
        & \beta' \leftarrow \mathcal{A}_2(state, C^*);
    \end{array}
    \right] - \frac{1}{2}.
\end{displaymath}
We say a PKE scheme is IND-CPA secure if no PPT adversary $\mathcal{A}$ 
has non-negligible advantage in the above security experiment.
\end{trivlist}



\begin{trivlist}
\item \textbf{Key Privacy/Anonymity.} Let $\mathcal{A}$ be an adversary against the anonymity of the PKE component 
and define its advantage in the following experiment:
\begin{displaymath}
    \AdvA = \Pr \left[\beta=\beta': 
    \begin{array}{ll}
        & pp \leftarrow \mathsf{Setup}(1^\lambda); \\
        & (pk_0, sk_0), (pk_1, sk_1) \leftarrow \mathsf{KeyGen}(pp)\\
        & (state, m) \leftarrow \mathcal{A}_1(pp, pk_1, pk_2);\\
        & \beta \sample \{0,1\};\\
        & C^* \leftarrow \mathsf{Enc}(pk_\beta, m)\\
        & \beta' \leftarrow \mathcal{A}_2(state, C^*);
    \end{array}
    \right] - \frac{1}{2}.
\end{displaymath}
We say a PKE scheme is anonymous (i.e., satisfying key privacy) if no PPT adversary $\mathcal{A}$ 
has non-negligible advantage in the above security experiment.
\end{trivlist}  

\section{Scheme Description and Security Analysis}
\subsection{Twisted ElGamal} 
Twisted ElGamal consists of four algorithms as below: 

\begin{trivlist}
\item Basic algorithms:
\begin{itemize} \itemsep 1pt \parskip 0pt \parsep 0pt
\item $\mathsf{Setup}(1^\lambda)$: run $(\mathbb{G}, g, p) \leftarrow \mathsf{GroupGen}(1^\lambda)$, 
    pick $h \sample \mathbb{G}^*$, set $pp = (\mathbb{G}, g, h, p)$ as global public parameters. 
    The randomness and message spaces are $\mathbb{Z}_p$. 

\item $\mathsf{KeyGen}(pp)$: on input $pp$, choose $sk \sample \mathbb{Z}_p$, set $pk = g^{sk}$. 

\item $\mathsf{Enc}(pk, m; r)$: compute $X = pk^r$, $Y = g^r h^m$, output $C = (X, Y)$. 

\item $\mathsf{Dec}(sk, C)$: parse $C = (X, Y)$, compute $h^m = Y/X^{sk^{-1}}$, recover $m$ from $h^m$. 
\end{itemize} 

\item Extended algorithms:
\begin{itemize} \itemsep 1pt \parskip 0pt \parsep 0pt
    \item $\mathsf{ReRand}(sk, C; r')$: on input $sk$, $C = (X, Y)$ and randomness $r'$, 
        compute $h^m = Y/X^{sk^{-1}}$, then compute $X' = pk^{r'}$, $Y' = g^{r'}h^m$, output $C' = (X', Y')$. 

    \item $\mathsf{HomoAdd}(C_1, C_2)$: on input $C_1 = (X_1, Y_1)$ and $C_2 = (X_2, Y_2)$, 
        compute $X = X_1+X_2$, $Y = Y_1+Y_2$, output $C = (X, Y)$.  

    \item $\mathsf{HomoSub}(C_1, C_2)$: on input $C_1 = (X_1, Y_1)$ and $C_2 = (X_2, Y_2)$, 
        compute $X = X_1-X_2$, $Y = Y_1-Y_2$, output $C = (X, Y)$.  

    \item $\mathsf{HomoScalar}(C, k)$: on input $C$ and a scalar $k \in \mathbb{Z}_p$, 
        compute $X' = k \cdot X$, $Y' = k \cdot Y$, output $C' = (X', Y')$.  
\end{itemize}
\end{trivlist}

Correctness and additive homomorphism are obvious. 
The IND-CPA security and anonymity are established by the following two theorems. 

\begin{theorem}\label{theorem:IND-CPA}
Twisted ElGamal is IND-CPA secure based on the divisible DDH assumption. 
\end{theorem}
\begin{proof}
We proceed via two games. Let $S_i$ be the probability that $\mathcal{A}$ wins in Game $i$. 
\begin{trivlist}
\item $\textbf{Game 0.}$ The real IND-CPA security experiment. 
Challenger $\mathcal{CH}$ interacts with $\mathcal{A}$ as below: 
\begin{enumerate}
\item \underline{Setup:} $\mathcal{CH}$ generates $pp$ and keypair as per definition, 
    then sends $pp$ and $pk = g^{sk}$ to $\mathcal{A}$. 

\item \underline{Challenge:} $\mathcal{A}$ submits $m_0, m_1$ to $\mathcal{CH}$ as the target messages. 
    $\mathcal{CH}$ picks a random bit $\beta$ 
    and randomness $r$, computes $X = pk^r$, $Y = g^r h^{m_\beta}$, sends $C = (X, Y)$ to $\mathcal{A}$.   

\item \underline{Guess:} $\mathcal{A}$ outputs its guess $\beta'$ for $\beta$ and wins if $\beta' = \beta$. 
\end{enumerate}

\item According to the definition of Game 0, we have: 
\begin{equation*}
    \mathsf{Adv}_\mathcal{A}(\lambda) = \Pr[S_0] - 1/2.
\end{equation*}
\end{trivlist}

\begin{trivlist}
\item $\textbf{Game 1.}$ Same as Game 0 except $\mathcal{CH}$ generates the challenge ciphertext in a different way.  

\begin{enumerate}\setcounter{enumi}{1}
\item \underline{Challenge:} $\mathcal{A}$ submits $m_0, m_1$ to $\mathcal{CH}$ as the target messages. 
    $\mathcal{CH}$ picks a random bit $\beta$ and \redul{two independent randomness $r$ and $s$, 
    computes $X = pk^r$, $Y = g^s h^{m_\beta}$}, sends $C = (X, Y)$ to $\mathcal{A}$.   
\end{enumerate}

\item In Game 1, the distribution of $C$ is independent of $\beta$, thus we have:
\begin{equation*}
    \Pr[S_1] = 1/2.
\end{equation*}
\end{trivlist}

It remains to prove $\Pr[S_1]$ and $\Pr[S_0]$ are negligibly close. 
We prove this by showing if not so, one can build an adversary $\mathcal{B}$ breaks the divisible DDH assumption 
with the same advantage. Given $(g, g^a, g^b, g^c)$, 
$\mathcal{B}$ is asked to decide if it is a divisible DDH tuple or a random tuple. 
To do so, $\mathcal{B}$ interacts with $\mathcal{A}$ by simulating $\mathcal{A}$'s challenger 
in the following IND-CPA experiment. 
\begin{enumerate}
\item \underline{Setup:} $\mathcal{B}$ picks $h \sample \mathbb{G}^*$, sets $pp = (\mathbb{G}, g, h, p)$, 
    sets $g^b$ as the public key, where $b \in \mathbb{Z}_p$ is interpreted as the corresponding secret key, 
    which is unknown to $\mathcal{B}$. $\mathcal{B}$ sends $pp$ and $pk = g^b$ to $\mathcal{A}$. 

\item \underline{Challenge:} $\mathcal{A}$ submits $m_0, m_1$ to $\mathcal{B}$. 
    $\mathcal{B}$ picks a random bit $\beta$ 
    and sets $X = g^a$, $Y = g^c h^{m_\beta}$, sends $C = (X, Y)$ to $\mathcal{A}$.  

\item \underline{Guess:} $\mathcal{A}$ outputs a guess $\beta'$. 
    $\mathcal{B}$ outputs ``1'' if $\beta' = \beta$ and ``0'' otherwise.  
\end{enumerate}   
If $(g, g^a, g^b, g^c)$ is a divisible DDH tuple, 
$\mathcal{B}$ simulates Game 0 perfectly (with randomness $c = a/b$).  
Else, $\mathcal{B}$ simulates Game 1 perfectly (with two independent randomness $a/b$ and $c$). 
Thereby, we have $\mathsf{Adv}_\mathcal{B} = |\Pr[S_0] - \Pr[S_1]|$, 
which is negligible in $\lambda$ by the divisible DDH assumption. 

Putting all the above together, Theorem~\ref{theorem:IND-CPA} follows.  
\end{proof}

\begin{remark}
We stress that twisted ElGamal remains secure even the adversary knows the discrete relation between $(g, h)$. 
Thereby, $h$ can be set as a fixed or random element in $\mathbb{G}^*$.   
In both cases, twisted ElGamal does not require trusted setup. 
Looking ahead, we set $h$ as a hash value of $g$ in our implementation, 
which is in line with the transparent setup in Bulletproofs~\cite{BulletProof2018}.   
\end{remark}

\begin{table}[!hbth]
\begin{center}
\begin{tabular}{|c|c|c|c|c|c|c|c|}
\hline
& \multicolumn{4}{c|}{size} & \multicolumn{3}{c|}{efficiency}\\
\hline
    & $pp$ & $pk$ & $sk$ & $C$ & $\mathsf{KeyGen}$ & $\mathsf{Enc}$ & $\mathsf{Dec}$\\
\hline
standard ElGamal & $|\mathbb{G}|$ & $|\mathbb{G}|$ & $|\mathbb{Z}_p|$ & $|2\mathbb{G}|$ 
                 & 1Exp & 3Exp+2Add & 1Exp+1Add+1DLOG\\
\hline
twisted ElGamal & \red{$2|\mathbb{G}|$} & $|\mathbb{G}|$ & $|\mathbb{Z}_p|$ & $|2\mathbb{G}|$ 
                 & 1Exp & 3Exp+2Add & 1Exp+1Add+1DLOG\\
\hline
\end{tabular}{}\label{table:comparison}
\caption{Comparison between standard ElGamal and twisted ElGamal. 
$|\mathbb{Z}_p|$ denotes the bits of a scalar field element, $|\mathbb{G}|$ denotes the bits of a group element. 
Exp and Add denote exponentiation and add operation in $\mathbb{G}$, while DLOG denotes discrete search operation in $\mathbb{G}$.}
\end{center}
\end{table}

\begin{theorem}\label{theorem:anonymity}
Twisted ElGamal is anonymous based on the divisible DDH assumption. 
\end{theorem}
\begin{proof}
We proceed via four games. 
\begin{trivlist}
\item $\textbf{Game 0.}$ Challenger $\mathcal{CH}$ interacts with $\mathcal{A}$ as follows: 
\begin{enumerate}
\item \underline{Setup:} $\mathcal{CH}$ generates $pp$ and two keypairs as per definition, 
    then sends $pp$ and $pk_1$ and $pk_2$ to $\mathcal{A}$. 

\item \underline{Challenge:} $\mathcal{A}$ submits $m$ to $\mathcal{CH}$ as the target message. 
    $\mathcal{CH}$ picks a randomness $r$, computes $X = pk_1^r$, $Y = g^r h^m$, 
    sends $C = (X, Y)$ to $\mathcal{A}$.   

\item \underline{Guess:} $\mathcal{A}$ outputs its guess $\beta'$ for $\beta$ and wins if $\beta' = \beta$. 
\end{enumerate}
\end{trivlist}

\begin{trivlist}
\item $\textbf{Game 1.}$ Same as Game 0 except $\mathcal{CH}$ generates the challenge ciphertext in a different way.  

\begin{enumerate}\setcounter{enumi}{1}
\item \underline{Challenge:} $\mathcal{A}$ submits $m$ to $\mathcal{CH}$ as the target message. 
    $\mathcal{CH}$ picks a random bit $\beta$ and \redul{two independent randomness $r$ and $s$, 
    computes $X = pk_1^r$, $Y = g^s h^m$}, sends $C = (X, Y)$ to $\mathcal{A}$.   
\end{enumerate}
\end{trivlist}

\begin{trivlist}
\item $\textbf{Game 2.}$ Same as Game 1 except $\mathcal{CH}$ generates ciphertext using $pk_2$.  

\begin{enumerate}\setcounter{enumi}{1}
\item \underline{Challenge:} $\mathcal{A}$ submits $m$ to $\mathcal{CH}$ as the target message. 
    $\mathcal{CH}$ picks a random bit $\beta$ and \redul{two independent randomness $r$ and $s$, 
    computes $X = pk_2^r$, $Y = g^s h^m$}, sends $C = (X, Y)$ to $\mathcal{A}$.   
\end{enumerate}
\end{trivlist}

\begin{trivlist}
\item $\textbf{Game 3.}$ Same as Game 2 except $\mathcal{CH}$ generates ciphertext under the same randomness.  

\begin{enumerate}\setcounter{enumi}{1}
\item \underline{Challenge:} $\mathcal{A}$ submits $m$ to $\mathcal{CH}$ as the target message. 
    $\mathcal{CH}$ picks randomness $r$, computes $X = pk_2^r$, $Y = g^r h^m$, sends $C = (X, Y)$ to $\mathcal{A}$.   
\end{enumerate}
\end{trivlist}

It is easy to see that Game 0 $\approx_c$ Game 1, Game 2 $\approx_c$ Game 3 
are computationally indistinguishable based on the divisible DDH assumption, while Game 1 $\equiv$ Game 2. 
Putting all the above together, Theorem~\ref{theorem:anonymity} follows.  
\end{proof}

Recall that the divisible DDH assumption is equivalent to the standard DDH assumption, 
thus we conclude that twisted ElGamal is IND-CPA secure and anonymous based on the standard DDH assumption. 
Moreover, both of the security reductions are tight. 



\section{Optimizations and Discussions}\label{sec:discussions}
In this section, we describe optimizations and summarize the advantages of twisted ElGamal. 

\subsection{Faster Decryption Algorithm}
As in standard exponential ElGamal, to ensure additive homomorphism on message space $\mathbb{Z}_p$, 
message in twisted ElGamal is also encoded to the exponent. 
This makes decryption inefficient even with knowledge of secret key. 
The bottleneck is the interval-version discrete logarithm search algorithm. 
Let $[0, 2^\ell)$ be the interval (message space).  
The time complexity of brute-force is $O(2^\ell)$.       
We recommend employing dedicated algorithms to compute a discrete logarithm in an interval more efficiently. 
The candidate algorithms include: 
\begin{itemize} \itemsep 1pt \parskip 0pt \parsep 0pt
    \item Pollard's kangaroo method~\cite{Pollard1978} and Galbraith-Ruprai algorithm~\cite{GR2010-PKC}: 
        run in time $O(2^{\ell/2})$ with constant memory cost. 
    
    \item Bernstein-Lange algorithm~\cite{BL2012}: runs in time $O(2^{\ell/3})$ using a table of size $O(2^{\ell/3})$.

    \item Shanks's algorithm~\cite{Shanks1971}: runs in time $O(2^{\ell/2})$ using a table of size $O(2^{\ell/2})$.   
\end{itemize} 
In this submission, we choose Shanks's algorithm for the following reasons: 
(i) the first two algorithms run in expected polynomial time, namely, 
they may take long time to find the solution in worst cases; 
(ii) Shanks's algorithm is simple and admits flexible time-space trade off. 
Let $n$ be an integer between $[0, \ell/2]$, it runs in time $O(2^{\ell/2-n})$ using a table of size $O(2^{\ell/2+n})$.
We refer to $n$ as the time-space tuning parameter, which is set to be $0$ in the standard Shanks's algorithm.  
% When $\ell = 32$ and $n = 7$, the average decryption time is less than 2ms by using a table of roughly 280MB. 
% This demonstrates that Shanks's algorithm is more preferable in practice.


\subsection{Signature and Zero-Knowledge Friendly}
In many privacy-preserving applications, homomorphic PKE is used in combination with other cryptographic schemes. 
As we will show, twisted ElGamal enjoys nice interoperability with signature and zero-knowledge proofs.   
\begin{trivlist}
\item \textbf{Integrate with signature.} The keypair of twisted ElGamal is a discrete logarithm tuple, 
    which is identical to widely used signature in practice, such as ECDSA and Schnorr signature.
    This fact enables efficient design of integrated signature and encryption schemes 
    by composing twisted ElGamal and ECDSA/Schnorr. 
    Please refer to~\cite{CMT2019} for more technical details, include formal security proof and example application. 

\item \textbf{Work with zero-knowledge proofs.} The algebra structure of twisted ElGamal 
    admits simple and efficient Sigma protocols 
for proving knowledge of plaintext and randomness, two ciphertexts encrypt the same message. 
Moreover, we can design efficient accompanying range proofs 
by directly invoking the state-of-the-art Bulletproofs~\cite{BulletProof2018}.  
This is superior than the standard ElGamal PKE. 
To see why, recall that Bulletproofs only accepts statements of the form of Pedersen commitment $g^r h^m$.  
To guarantee soundness, the DL relation between commitment key $(g, h)$ must be unknown to the prover. 
However, a standard ElGamal ciphertext $C$ of $m$ under $pk$ is of the form $(g^r, pk^r g^m)$. 
Although the second component can be viewed as a commitment of $m$ under commitment key $(pk, g)$,   
we cannot \emph{directly} employ Bulletproofs to generate range proof for $m$, because $sk$ constitutes an obvious trapdoor. 
In contrast, the second component of twisted ElGamal ciphertext is exactly of the form of Pedersen commitment $g^r h^m$, 
and the DL relation between $(g, h)$ is unknown. 
This allows us to directly invoke Bulletproofs to generate range proofs for encrypted values under twisted ElGamal 
in a black-box manner, and these proofs can be easily aggregated. 
Please refer to~\cite{CMT2019} for more technical details, including useful gadgets and formal security proof. 

We note that we do can employ Bulletproofs to generate range proof for messages encrypted by standard ElGamal PKE. 
But, we need to use a Pedersen commitment as a bridge, 
i.e., (1) generate a Pedersen commitment $g^r h^m$; (2) prove the randomness and message under ElGamal ciphertext $(g^r, pk^r h^m)$ 
and Pedersen commitment $g^r h^m$ are consistent via a Sigma protocol; (3) invoke Bulletproofs on $g^r h^m$. 
By using twisted ElGamal in the place of standard ElGamal, we are able to dismiss steps (1) and (2).  
We compare the cost of working with Bulletproofs between standard and twisted ElGamal PKE in the following table. 
\end{trivlist}

\begin{table}[H]
\begin{center}
\begin{tabular}{|c|c|c|}
\hline
& size & efficiency\\
\hline
standard ElGamal & $2|\mathbb{G}|+|\mathbb{Z}_p|$ & 4Exp+1Add\\
\hline
twisted ElGamal & \blue{0} & \blue{0}\\
\hline
\end{tabular}{}\label{table:comparison}
\caption{The costs of working with Bulletproofs between standard ElGamal and twisted ElGamal. 
The cost of ElGamal working with Bulletproofs comes from two parts, 
i.e., an additional Pedersen commitment and a Sigma protocol for consistency between Pedersen commitment and ElGamal ciphertext.}
\end{center}
\end{table}


\section{Implementation and Evaluation}
We report an implementation of twisted ElGamal in C++ based on OpenSSL, 
and collect the benchmarks on a MacBook Pro with an Intel i7-4870HQ CPU 
(2.5GHz) and 16GB of RAM, using elliptic curve NID\_X9\_62\_prime256v1\footnote{Our implementation also supports other elliptic curves.}~\cite{OpenSSL_curve}, which has 128 bit security. 
The default message space is 40bit, i.e., $[0, 2^{40}-1)$. 
$\mathsf{Enc}$ and $\mathsf{Scalar}$ in 2 thread setting are roughly 15\% $\sim$ 25\% 
faster than their counterpart in the single thread setting. 
\begin{table}[!hbth]
\begin{center}
\begin{tabular}{|c|c|c|c|c|c|c|c|c|c|}
\hline
& hash map & $\mathsf{Setup}$ & $\mathsf{KeyGen}$ & $\mathsf{Enc}$ & $\mathsf{Dec}$ & 
  $\mathsf{ReRand}$ & $\mathsf{Add}$ & $\mathsf{Sub}$ & $\mathsf{Scalar}$ \\
\hline
\multirow{2}{*}{1 thread}
   & 275MB & \red{50s+6s} & 0.0158 & 0.141 & 400 & 0.157 & 0.0031 & 0.0039 & 0.136\\
\cline{2-10}
   & 550MB & \red{100s+12s} & ---     & --- & 200 & ---   & ---    & ---    & ---  \\
\hline
\blue{2 thread} & --- & ---        & 0.116  & --- & 331   & ---   & ---    & ---    & 0.094\\
\hline
\blue{4 thread} & 275MB & ---        & ---     & --- & 173   & ---   & ---    & ---    & ---\\
\hline
\blue{4 thread} & 1.1GB & ---        & ---     & --- & 42   & ---   & ---    & ---    & ---\\
\hline
\end{tabular}{}\label{table:efficiency of twisted ElGamal}
\end{center}
\caption{Except the $\mathsf{Setup}$ algorithm (highlighted in red color), 
the efficiency of other algorithms are measured by ms. 
The decryption time is obtained using a hash map of size roughly 270MB. 
Faster decryption speed can be achieved by enlarging the table. 
The setup algorithm runs ``one-and-for-all'', 
it first generates and serializes a key-value hash map called ``\texttt{point2index.map}'' 
to accelerate decryption (this step takes roughly \red{50s}), 
then loads it to the memory (this step takes roughly \red{6s}).}
\end{table}



\bibliographystyle{alpha}
\bibliography{\path/mycrypto}

\newpage





 
\end{document}